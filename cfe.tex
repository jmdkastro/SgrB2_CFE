\documentclass[twocolumn]{aastex62}
%\documentclass[defaultstyle,11pt]{thesis}
%\documentclass[]{report}
%\documentclass[]{article}
%\usepackage{aastex_hack}
%\usepackage{deluxetable}
%\documentclass[preprint]{aastex}
%\documentclass{aa}

\newcommand{\titlerunning}[1]{\shorttitle{#1}}
\newcommand{\authorrunning}[1]{\shortauthors{#1}}

\newcommand*\inst[1]{\unskip\hbox{\@textsuperscript{\normalfont$#1$}}}

%\newcount\aa@nbinstitutes
%
%\newcounter{aa@institutecnt}

\newcommand*\institute[1]{
  \begingroup
    \let\and\relax
    \renewcommand*\inst[1]{}%
    \renewcommand*\thanks[1]{}%
    \renewcommand*\email[1]{}%
    %\let\@@protect\protect
    %\let\protect\@unexpandable@protect
    %\global\aa@nbinstitutes \z@
    %\expandafter\aa@cntinstitutes\aa@institute\and\aa@nil\and
    %\restore@protect
  \endgroup
  \newcommand{\institutions}{#1}
}%

\let\oldarcsec\arcsec
\renewcommand\arcsec{\oldarcsec\xspace}%

%\renewcommand{\abstract}[1]{
%\begin{abstract}
%    #1
%\end{abstract}
%}

%\renewcommand\ion[2]{#1$\;${%
%\ifx\@currsize\normalsize\small \else
%\ifx\@currsize\small\footnotesize \else
%\ifx\@currsize\footnotesize\scriptsize \else
%\ifx\@currsize\scriptsize\tiny \else
%\ifx\@currsize\large\normalsize \else
%\ifx\@currsize\Large\large
%\fi\fi\fi\fi\fi\fi
%\rmfamily\@Roman{#2}}\relax}% 
%
%\renewcommand{ion}[2]{#1}{#2}

\renewcommand{\ion}[2]{\textup{#1\,\textsc{\lowercase{#2}}}}

%\newcommand{\uchii}{\ensuremath{\mathrm{\ion{UCH}{2}}}\xspace}
%\newcommand{\UCHII}{\ensuremath{\mathrm{\ion{UCH}{2}}}\xspace}
%\newcommand{\hchii}{\ensuremath{\mathrm{\ion{HCH}{2}}}\xspace}
%\newcommand{\HCHII}{\ensuremath{\mathrm{\ion{HCH}{2}}}\xspace}
%\newcommand{\hii}  {\ensuremath{\mathrm{\ion{H}{2}}}\xspace}

%\input{aamacros.tex}

\pdfminorversion=4


%%%%%%%%%%%%%%%%%%%%%%%%%%%%%%%%%%%%%%%%%%%%%%%%%%%%%%%%%%%%%%%%
%%%%%%%%%%%  see documentation for information about  %%%%%%%%%%
%%%%%%%%%%%  the options (11pt, defaultstyle, etc.)   %%%%%%%%%%
%%%%%%%  http://www.colorado.edu/its/docs/latex/thesis/  %%%%%%%
%%%%%%%%%%%%%%%%%%%%%%%%%%%%%%%%%%%%%%%%%%%%%%%%%%%%%%%%%%%%%%%%
%		\documentclass[typewriterstyle]{thesis}
% 		\documentclass[modernstyle]{thesis}
% 		\documentclass[modernstyle,11pt]{thesis}
%	 	\documentclass[modernstyle,12pt]{thesis}

%%%%%%%%%%%%%%%%%%%%%%%%%%%%%%%%%%%%%%%%%%%%%%%%%%%%%%%%%%%%%%%%
%%%%%%%%%%%    load any packages which are needed    %%%%%%%%%%%
%%%%%%%%%%%%%%%%%%%%%%%%%%%%%%%%%%%%%%%%%%%%%%%%%%%%%%%%%%%%%%%%
\usepackage{latexsym}		% to get LASY symbols
\usepackage{graphicx}		% to insert PostScript figures
%\usepackage{deluxetable}
\usepackage{rotating}		% for sideways tables/figures
\usepackage{natbib}  % Requires natbib.sty, available from http://ads.harvard.edu/pubs/bibtex/astronat/
\usepackage{savesym}
%\usepackage{pdflscape}
\usepackage{amssymb}
\usepackage{amsmath}
\usepackage{morefloats}
%\savesymbol{singlespace}
\savesymbol{doublespace}
%\usepackage{wrapfig}
%\usepackage{setspace}
\usepackage{xspace}
\usepackage{color}
%\usepackage{multicol}
\usepackage{mdframed}
\usepackage{url}
\usepackage{subfigure}
%\usepackage{emulateapj}
%\usepackage{lscape}
\usepackage{grffile}
\usepackage{import}
\usepackage[utf8]{inputenc}
%\usepackage{longtable}
\usepackage{booktabs}
%\usepackage[yyyymmdd,hhmmss]{datetime}
\usepackage{fancyhdr}
%\usepackage[colorlinks=true,citecolor=blue,linkcolor=cyan]{hyperref}

\usepackage[hang,flushmargin]{footmisc}
\usepackage{ifpdf}
\usepackage{microtype} % just to use \textls to condense an affiliations into one line

\usepackage{times}

%\usepackage{standalone}
%\standalonetrue






\newcommand{\paa}{Pa\ensuremath{\alpha}}
\newcommand{\brg}{Br\ensuremath{\gamma}}
\newcommand{\msun}{\ensuremath{\mathrm{M}_{\odot}}\xspace}			%  Msun
\newcommand{\mdot}{\ensuremath{\dot{M}}\xspace}
\newcommand{\lsun}{\ensuremath{L_{\odot}}\xspace}			%  Lsun
\newcommand{\rsun}{\ensuremath{R_{\odot}}\xspace}			%  Rsun
\newcommand{\lbol}{\ensuremath{L_{\mathrm{bol}}\xspace}}	%  Lbol
\newcommand{\ks}{K\ensuremath{_{\mathrm{s}}}}		%  Ks
\newcommand{\hh}{\ensuremath{\textrm{H}_{2}}\xspace}			%  H2
\newcommand{\dens}{\ensuremath{n(\hh) [\percc]}\xspace}
\newcommand{\formaldehyde}{\ensuremath{\textrm{H}_2\textrm{CO}}\xspace}
\newcommand{\formamide}{\ensuremath{\textrm{NH}_2\textrm{CHO}}\xspace}
\newcommand{\formaldehydeIso}{\ensuremath{\textrm{H}_2~^{13}\textrm{CO}}\xspace}
\newcommand{\methanol}{\ensuremath{\textrm{CH}_3\textrm{OH}}\xspace}
\newcommand{\ortho}{\ensuremath{\textrm{o-H}_2\textrm{CO}}\xspace}
\newcommand{\para}{\ensuremath{\textrm{p-H}_2\textrm{CO}}\xspace}
\newcommand{\oneone}{\ensuremath{1_{1,0}-1_{1,1}}\xspace}
\newcommand{\twotwo}{\ensuremath{2_{1,1}-2_{1,2}}\xspace}
\newcommand{\threethree}{\ensuremath{3_{1,2}-3_{1,3}}\xspace}
\newcommand{\threeohthree}{\ensuremath{3_{0,3}-2_{0,2}}\xspace}
\newcommand{\threetwotwo}{\ensuremath{3_{2,2}-2_{2,1}}\xspace}
\newcommand{\threetwoone}{\ensuremath{3_{2,1}-2_{2,0}}\xspace}
\newcommand{\fourtwotwo}{\ensuremath{4_{2,2}-3_{1,2}}\xspace} % CH3OH 218.4 GHz
\newcommand{\methylcyanide}{\ensuremath{\textrm{CH}_{3}\textrm{CN}}\xspace}
\newcommand{\ketene}{\ensuremath{\textrm{H}_{2}\textrm{CCO}}\xspace}
\newcommand{\ethylcyanide}{\ensuremath{\textrm{CH}_3\textrm{CH}_2\textrm{CN}}\xspace}
\newcommand{\cyanoacetylene}{\ensuremath{\textrm{HC}_{3}\textrm{N}}\xspace}
\newcommand{\methylformate}{\ensuremath{\textrm{CH}_{3}\textrm{OCHO}}\xspace}
\newcommand{\dimethylether}{\ensuremath{\textrm{CH}_{3}\textrm{OCH}_{3}}\xspace}
\newcommand{\gaucheethanol}{\ensuremath{\textrm{g-CH}_3\textrm{CH}_2\textrm{OH}}\xspace}
\newcommand{\acetone}{\ensuremath{\left[\textrm{CH}_{3}\right]_2\textrm{CO}}\xspace}
\newcommand{\methyleneamidogen}{\ensuremath{\textrm{H}_{2}\textrm{CN}}\xspace}
\newcommand{\Rone}{\ensuremath{\para~S_{\nu}(\threetwoone) / S_{\nu}(\threeohthree)}\xspace}
\newcommand{\Rtwo}{\ensuremath{\para~S_{\nu}(\threetwotwo) / S_{\nu}(\threetwoone)}\xspace}
\newcommand{\JKaKc}{\ensuremath{J_{K_a K_c}}}
\newcommand{\water}{H$_{2}$O\xspace}		%  H2O
\newcommand{\feii}{\ion{Fe}{ii}\xspace}		%  FeII
\newcommand{\pc}{\ensuremath{\mathrm{pc}}\xspace}
\newcommand{\myr}{\ensuremath{\mathrm{Myr}}\xspace}

\newcommand{\uchii}{\ion{UCH}{ii}\xspace}
\newcommand{\UCHII}{\ion{UCH}{ii}\xspace}
\newcommand{\hchii}{\ion{HCH}{ii}\xspace}
\newcommand{\HCHII}{\ion{HCH}{ii}\xspace}
\newcommand{\hii}{\ion{H}{ii}\xspace}

\newcommand{\hi}{H~{\sc i}\xspace}
\newcommand{\Hii}{\hii}
\newcommand{\HII}{\hii}
\newcommand{\Xform}{\ensuremath{X_{\formaldehyde}}}
\newcommand{\kms}{\textrm{km~s}\ensuremath{^{-1}}\xspace}	%  km s-1
\newcommand{\nsample}{456\xspace}
\newcommand{\CFR}{5\xspace} % nMPC / 0.25 / 2 (6 for W51 once, 8 for W51 twice) REFEDIT: With f_observed=0.3, becomes 3/2./0.3 = 5
\newcommand{\permyr}{\ensuremath{\mathrm{Myr}^{-1}}\xspace}
\newcommand{\pers}{\ensuremath{\mathrm{s}^{-1}}\xspace}
\newcommand{\perspc}{\ensuremath{\mathrm{pc}^{-2}}\xspace}
\newcommand{\tsuplim}{0.5\xspace} % upper limit on starless timescale
\newcommand{\ncandidates}{18\xspace}
\newcommand{\mindist}{8.7\xspace}
\newcommand{\rcluster}{2.5\xspace}
\newcommand{\ncomplete}{13\xspace}
\newcommand{\middistcut}{13.0\xspace}
\newcommand{\nMPC}{3\xspace} % only count W51 once.  W51, W49, G010
\newcommand{\obsfrac}{30}
\newcommand{\nMPCtot}{10\xspace} % = nmpc / obsfrac
\newcommand{\nMPCtoterr}{6\xspace} % = sqrt(nmpc) / obsfrac
\newcommand{\plaw}{2.1\xspace}
\newcommand{\plawerr}{0.3\xspace}
\newcommand{\mmin}{\ensuremath{10^4~\msun}\xspace}
%\newcommand{\perkmspc}{\textrm{per~km~s}\ensuremath{^{-1}}\textrm{pc}\ensuremath{^{-1}}\xspace}	%  km s-1 pc-1
\newcommand{\kmspc}{\textrm{km~s}\ensuremath{^{-1}}\textrm{pc}\ensuremath{^{-1}}\xspace}	%  km s-1 pc-1
\newcommand{\sqcm}{cm$^{2}$\xspace}		%  cm^2
\newcommand{\percc}{\ensuremath{\textrm{cm}^{-3}}\xspace}
\newcommand{\perpc}{\ensuremath{\textrm{pc}^{-1}}\xspace}
\newcommand{\persc}{\ensuremath{\textrm{cm}^{-2}}\xspace}
\newcommand{\persr}{\ensuremath{\textrm{sr}^{-1}}\xspace}
\newcommand{\peryr}{\ensuremath{\textrm{yr}^{-1}}\xspace}
\newcommand{\perkmspc}{\textrm{km~s}\ensuremath{^{-1}}\textrm{pc}\ensuremath{^{-1}}\xspace}	%  km s-1 pc-1
\newcommand{\perkms}{\textrm{per~km~s}\ensuremath{^{-1}}\xspace}	%  km s-1 
\newcommand{\um}{\ensuremath{\mu \textrm{m}}\xspace}    % micron
\newcommand{\microjy}{\ensuremath{\mu\textrm{Jy}}\xspace}    % micron
\newcommand{\mum}{\um}
\newcommand{\htwo}{\ensuremath{\textrm{H}_2}}
\newcommand{\Htwo}{\ensuremath{\textrm{H}_2}}
\newcommand{\HtwoO}{\ensuremath{\textrm{H}_2\textrm{O}}}
\newcommand{\htwoo}{\ensuremath{\textrm{H}_2\textrm{O}}}
\newcommand{\ha}{\ensuremath{\textrm{H}\alpha}}
\newcommand{\hb}{\ensuremath{\textrm{H}\beta}}
\newcommand{\so}{SO~\ensuremath{5_6-4_5}\xspace}
\newcommand{\SO}{SO~\ensuremath{1_2-1_1}\xspace}
\newcommand{\ammonia}{NH\ensuremath{_3}\xspace}
\newcommand{\twelveco}{\ensuremath{^{12}\textrm{CO}}\xspace}
\newcommand{\thirteenco}{\ensuremath{^{13}\textrm{CO}}\xspace}
\newcommand{\ceighteeno}{\ensuremath{\textrm{C}^{18}\textrm{O}}\xspace}
\def\ee#1{\ensuremath{\times10^{#1}}}
\newcommand{\degrees}{\ensuremath{^{\circ}}}
% can't have \degree because I'm getting a degree...
\newcommand{\lowirac}{800}
\newcommand{\highirac}{8000}
\newcommand{\lowmips}{600}
\newcommand{\highmips}{5000}
\newcommand{\perbeam}{\ensuremath{\textrm{beam}^{-1}}\xspace}
\newcommand{\ds}{\ensuremath{\textrm{d}s}}
\newcommand{\dnu}{\ensuremath{\textrm{d}\nu}}
\newcommand{\dv}{\ensuremath{\textrm{d}v}}
\def\secref#1{Section \ref{#1}}
\def\eqref#1{Equation \ref{#1}}
\def\facility#1{#1}
%\newcommand{\arcmin}{'}

\def\citenum#1{#1}

\newcommand{\necluster}{Sh~2-233IR~NE}
\newcommand{\swcluster}{Sh~2-233IR~SW}
\newcommand{\region}{IRAS 05358}

\newcommand{\nwfive}{40}
\newcommand{\nouter}{15}

\newcommand{\vone}{{\rm v}1.0\xspace}
\newcommand{\vtwo}{{\rm v}2.0\xspace}
\newcommand\mjysr{\ensuremath{{\rm MJy~sr}^{-1}}}
\newcommand\jybm{\ensuremath{{\rm Jy~bm}^{-1}}}
\newcommand\nbolocat{8552\xspace}
\newcommand\nbolocatnew{548\xspace}
\newcommand\nbolocatnonew{8004\xspace} % = nbolocat-nbolocatnew
%\renewcommand\arcdeg{\mbox{$^\circ$}\xspace} 
%\renewcommand\arcmin{\mbox{$^\prime$}\xspace} 
%\renewcommand\arcsec{\mbox{$^{\prime\prime}$}\xspace} 

\newcommand{\todo}[1]{\textcolor{red}{#1}}
\newcommand{\okinfinal}[1]{{#1}}
%% only needed if not aastex
%\newcommand{\keywords}[1]{}
%\newcommand{\email}[1]{}
%\newcommand{\affil}[1]{}


%aastex hack
%\newcommand\arcdeg{\mbox{$^\circ$}}%
%\newcommand\arcmin{\mbox{$^\prime$}\xspace}%
%\newcommand\arcsec{\mbox{$^{\prime\prime}$}\xspace}%

%\newcommand\epsscale[1]{\gdef\eps@scaling{#1}}
%
%\newcommand\plotone[1]{%
% \typeout{Plotone included the file #1}
% \centering
% \leavevmode
% \includegraphics[width={\eps@scaling\columnwidth}]{#1}%
%}%
%\newcommand\plottwo[2]{{%
% \typeout{Plottwo included the files #1 #2}
% \centering
% \leavevmode
% \columnwidth=.45\columnwidth
% \includegraphics[width={\eps@scaling\columnwidth}]{#1}%
% \hfil
% \includegraphics[width={\eps@scaling\columnwidth}]{#2}%
%}}%


%\newcommand\farcm{\mbox{$.\mkern-4mu^\prime$}}%
%\let\farcm\farcm
%\newcommand\farcs{\mbox{$.\!\!^{\prime\prime}$}}%
%\let\farcs\farcs
%\newcommand\fp{\mbox{$.\!\!^{\scriptscriptstyle\mathrm p}$}}%
%\newcommand\micron{\mbox{$\mu$m}}%
%\def\farcm{%
% \mbox{.\kern -0.7ex\raisebox{.9ex}{\scriptsize$\prime$}}%
%}%
%\def\farcs{%
% \mbox{%
%  \kern  0.13ex.%
%  \kern -0.95ex\raisebox{.9ex}{\scriptsize$\prime\prime$}%
%  \kern -0.1ex%
% }%
%}%

\def\Figure#1#2#3#4#5{
\begin{figure*}[!htp]
\includegraphics[scale=#4,width=#5]{#1}
\caption{#2}
\label{#3}
\end{figure*}
}

\def\FigureOneCol#1#2#3#4#5{
\begin{figure}[!htp]
\includegraphics[scale=#4,width=#5]{#1}
\caption{#2}
\label{#3}
\end{figure}
}


\def\WrapFigure#1#2#3#4#5#6{
\begin{wrapfigure}{#6}{0.5\textwidth}
\includegraphics[scale=#4,width=#5]{#1}
\caption{#2}
\label{#3}
\end{wrapfigure}
}

% % #1 - filename
% % #2 - caption
% % #3 - label
% % #4 - epsscale
% % #5 - R or L?
% \def\WrapFigure#1#2#3#4#5#6{
% \begin{wrapfigure}[#6]{#5}{0.45\textwidth}
% %  \centercaption
% %  \vspace{-14pt}
%   \epsscale{#4}
%   \includegraphics[scale=#4]{#1}
%   \caption{#2}
%   \label{#3}
% \end{wrapfigure}
% }

\def\RotFigure#1#2#3#4#5{
\begin{sidewaysfigure*}[!htp]
\includegraphics[scale=#4,width=#5]{#1}
\caption{#2}
\label{#3}
\end{sidewaysfigure*}
}

\def\FigureSVG#1#2#3#4{
\begin{figure*}[!htp]
    \def\svgwidth{#4}
    \input{#1}
    \caption{#2}
    \label{#3}
\end{figure*}
}

% originally intended to be included in a two-column paper
% this is in includegraphics: ,width=3in
% but, not for thesis
\def\OneColFigure#1#2#3#4#5{
\begin{figure}[!htpb]
\epsscale{#4}
\includegraphics[scale=#4,angle=#5]{#1}
\caption{#2}
\label{#3}
\end{figure}
}

\def\SubFigure#1#2#3#4#5{
\begin{figure*}[!htp]
\addtocounter{figure}{-1}
\epsscale{#4}
\includegraphics[angle=#5]{#1}
\caption{#2}
\label{#3}
\end{figure*}
}


\def\FigureTwo#1#2#3#4#5#6{
\begin{figure*}[!htp]
\subfigure[]{ \includegraphics[scale=#5,width=#6]{#1} }
\subfigure[]{ \includegraphics[scale=#5,width=#6]{#2} }
\caption{#3}
\label{#4}
\end{figure*}
}

\def\FigureTwoAA#1#2#3#4#5#6{
\begin{figure*}[!htp]
\subfigure[]{ \includegraphics[scale=#5,width=#6]{#1} }
\subfigure[]{ \includegraphics[scale=#5,width=#6]{#2} }
\caption{#3}
\label{#4}
\end{figure*}
}

\newenvironment{rotatepage}
{}{}


\def\RotFigureTwoAA#1#2#3#4#5#6{
\begin{rotatepage}
\begin{sidewaysfigure*}[!htp]
\subfigure[]{ \includegraphics[scale=#5,width=#6]{#1} }
\\
\subfigure[]{ \includegraphics[scale=#5,width=#6]{#2} }
\caption{#3}
\label{#4}
\end{sidewaysfigure*}
\end{rotatepage}
}

\def\RotFigureThreeAA#1#2#3#4#5#6#7{
\begin{rotatepage}
\begin{sidewaysfigure*}[!htp]
\subfigure[]{ \includegraphics[scale=#6,width=#7]{#1} }
\\
\subfigure[]{ \includegraphics[scale=#6,width=#7]{#2} }
\\
\subfigure[]{ \includegraphics[scale=#6,width=#7]{#3} }
\caption{#4}
\label{#5}
\end{sidewaysfigure*}
\end{rotatepage}
\clearpage
}

\def\FigureThreeAA#1#2#3#4#5#6#7{
\begin{figure*}[!htp]
\subfigure[]{ \includegraphics[scale=#6,width=#7]{#1} }
\subfigure[]{ \includegraphics[scale=#6,width=#7]{#2} }
\subfigure[]{ \includegraphics[scale=#6,width=#7]{#3} }
\caption{#4}
\label{#5}
\end{figure*}
}



\def\SubFigureTwo#1#2#3#4#5{
\begin{figure*}[!htp]
\addtocounter{figure}{-1}
\epsscale{#5}
\plottwo{#1}{#2}
\caption{#3}
\label{#4}
\end{figure*}
}

\def\FigureFour#1#2#3#4#5#6{
\begin{figure*}[!htp]
\subfigure[]{ \includegraphics[width=0.5\textwidth]{#1} }
\subfigure[]{ \includegraphics[width=0.5\textwidth]{#2} }
\subfigure[]{ \includegraphics[width=0.5\textwidth]{#3} }
\subfigure[]{ \includegraphics[width=0.5\textwidth]{#4} }
\caption{#5}
\label{#6}
\end{figure*}
}

\def\FigureFourPDF#1#2#3#4#5#6{
\begin{figure*}[!htp]
\subfigure[]{ \includegraphics[width=3in,type=pdf,ext=.pdf,read=.pdf]{#1} }
\subfigure[]{ \includegraphics[width=3in,type=pdf,ext=.pdf,read=.pdf]{#2} }
\subfigure[]{ \includegraphics[width=3in,type=pdf,ext=.pdf,read=.pdf]{#3} }
\subfigure[]{ \includegraphics[width=3in,type=pdf,ext=.pdf,read=.pdf]{#4} }
\caption{#5}
\label{#6}
\end{figure*}
}

\def\FigureThreePDF#1#2#3#4#5{
\begin{figure*}[!htp]
\subfigure[]{ \includegraphics[width=3in,type=pdf,ext=.pdf,read=.pdf]{#1} }
\subfigure[]{ \includegraphics[width=3in,type=pdf,ext=.pdf,read=.pdf]{#2} }
\subfigure[]{ \includegraphics[width=3in,type=pdf,ext=.pdf,read=.pdf]{#3} }
\caption{#4}
\label{#5}
\end{figure*}
}

\def\Table#1#2#3#4#5{
%\renewcommand{\thefootnote}{\alph{footnote}}
\begin{table}
\caption{#2}
\label{#3}
    \begin{tabular}{#1}
        \hline\hline
        #4
        \hline
        #5
        \hline
    \end{tabular}
\end{table}
%\renewcommand{\thefootnote}{\arabic{footnote}}
}


%\def\Table#1#2#3#4#5#6{
%%\renewcommand{\thefootnote}{\alph{footnote}}
%\begin{deluxetable}{#1}
%\tablewidth{0pt}
%\tabletypesize{\footnotesize}
%\tablecaption{#2}
%\tablehead{#3}
%\startdata
%\label{#4}
%#5
%\enddata
%\bigskip
%#6
%\end{deluxetable}
%%\renewcommand{\thefootnote}{\arabic{footnote}}
%}

%\def\tablenotetext#1#2{
%\footnotetext[#1]{#2}
%}

% \def\LongTable#1#2#3#4#5#6#7#8{
% % required to get tablenotemark to work: http://www2.astro.psu.edu/users/stark/research/psuthesis/longtable.html
% \renewcommand{\thefootnote}{\alph{footnote}}
% \begin{longtable}{#1}
% \caption[#2]{#2}
% \label{#4} \\
% 
%  \\
% \hline 
% #3 \\
% \hline
% \endfirsthead
% 
% \hline
% #3 \\
% \hline
% \endhead
% 
% \hline
% \multicolumn{#8}{r}{{Continued on next page}} \\
% \hline
% \endfoot
% 
% \hline 
% \endlastfoot
% #7 \\
% 
% #5
% \hline
% #6 \\
% 
% \end{longtable}
% \renewcommand{\thefootnote}{\arabic{footnote}}
% }

\def\TallFigureTwo#1#2#3#4#5#6{
\begin{figure*}[htp]
\epsscale{#5}
\subfigure[]{ \includegraphics[width=#6]{#1} }
\subfigure[]{ \includegraphics[width=#6]{#2} }
\caption{#3}
\label{#4}
\end{figure*}
}

		% file containing author's macro definitions
%%%% This file is generated by the Makefile.
\newcommand{\githash}{54b1618}\newcommand{\gitdate}{2018-05-09\xspace}\newcommand{\gitauthor}{Adam Ginsburg (keflavich)\xspace}


\begin{document}

%\title{AN EXTREMELY HIGH CLUSTER FORMATION EFFICIENCY IN THE MINI-STARBURST SGR B2}
\title{A high cluster formation efficiency in Sgr B2}

% Running title with <= 40-45 characters
\shorttitle{Cluster formation efficiency in Sgr~B2}
\shortauthors{Ginsburg \& Kruijssen}

\author{Adam Ginsburg}
\affiliation{National Radio Astronomy Observatory, 1003 Lopezville Rd., Socorro, NM 87801, USA}
\affiliation{Jansky Fellow}

\author{J.~M.\ Diederik Kruijssen}
\affiliation{\textls[-10]{Astronomisches Rechen-Institut, Zentrum f\"{u}r Astronomie der Universit\"{a}t Heidelberg, M\"{o}nchhofstra{\ss}e 12-14, D-69120 Heidelberg, Germany}}

\correspondingauthor{Adam Ginsburg}
\email{aginsbur@nrao.edu}
\email{adam.g.ginsburg@gmail.com}


\begin{abstract}
    The fraction of stars forming in dense, gravitationally bound clusters is
    an important parameter in understanding both the star formation history of
    the universe and the effects of stellar feedback from groups of stars.
    (maybe something about the K+ theory)
    We report a measurement of the cluster formation efficiency (CFE), the
    fraction of stars forming in clusters, in the highest-density region in
    the Galaxy, Sgr B2.  We find that about a third of the stars (37-43\%) in
    Sgr B2 are forming in bound clusters, a value consistent with the
    predictions of the \citet{Kruijssen2012a} models.
\end{abstract}

\section{Introduction}
Gravitationally bound stellar clusters are some of the most important objects
in astronomy, providing both luminous probes of the star formation process at
great distances and large coeval and co-located samples of stars in the local
universe.  The prevalence of these clusters varies substantially with
environment: the \emph{cluster formation efficiency} $\Gamma$ is not constant.

\citet{Kruijssen2012a} proposed a theory in which $\Gamma$ is a function of gas
density\footnote{In galactic disks, the relevant parameter is often gas surface
density, while in less ordered environments, it is the volume density}.  While
this theory reasonably explains observations spanning many galaxies, it has not
yet been directly tested in a high-density environment where both the
unclustered and clustered stars are detected.  
In this Letter, we perform such a test in the Sgr B2 cloud, a high-density
region in the Galactic center in which both stars and bound clusters are presently
forming.

% {\color{red}
% Notes to self:
% \begin{itemize}
%     %\item mass of stars, gas in M, N, distributed population, S? [mostly taken from Schmiedeke] [DONE?]
%     %\item virial parameters?  Use DePree's data for velocity dispersion?  [added my own H41a; DePree's are useful though] [ DONE - note to Diederik - this is for stars, not gas ]
%     %\item cluster definition: purely radial.  Is S a cluster?  Is NE?  [assuming "No" below; have added discussion about effect of expanding N]
%     \item Age difference.  What (maximal) errors can this impose?
% \end{itemize}
% }

\section{Observational Summary}

We use the catalogs described in \citet{Ginsburg2018a}, \citet{Gaume1995a}, and
\citet{De-Pree2015a} to infer the total stellar population.

\citet{Gaume1995a} observed Sgr B2 at 1.3 cm with $\sim0.25$\arcsec resolution
with the VLA.  They detected 49 continuum sources.  These are exclusively \hii
regions and components of \hii regions.  \citet{De-Pree2015a} used 7 mm JVLA
Q-band observations at 0.05\arcsec resolution to catalog 26 sources in Sgr B2 M
and 5 in Sgr B2 N.  Of these, 7 detected in Sgr B2M were not reported in
\citet{Gaume1995a} because they were not resolved.  We assume each of the
VLA-detected sources is an \hii region and therefore contains at least one star
with $M\gtrsim20$ \msun, equivalent to a B0 star.

\citet{Ginsburg2018a} observed the cloud with ALMA, obtaining a resolution of
$0.5\arcsec$ in the 3 mm band.  They reported a total of 271 sources spread
throughout the cloud, of which 31 are confirmed \hii regions with implied
masses $M_*>20$ \msun; the rest are young stellar object (YSO) candidates with
masses $M_*>8$ \msun.  We follow \citet{Ginsburg2018a} in extrapolating the total
stellar mass implied by the observed objects using a \citet{Kroupa2001a}
initial mass function.


\section{The mass of the clusters}
%Cluster masses are determined by counting the number of HII regions and high-mass
%protostellar cores associated with each of the clusters.

In Table 2 of \citet{Ginsburg2018a}, four clusters were considered: N, M, NE, and
S.  Here, we re-evaluate the ``clusters" in NE and S.  These regions contain
few sources and are not centrally concentrated.
They are both moderate mass and, at present, do not appear likely to form bound
clusters.  We therefore exclude them from the analysis, but note that if they
are forming bound clusters, the measured CFE would increase by a few percent.
% {\color{red} or, do with/without
% them}.

\begin{table*}[htp]
\centering
\begin{minipage}{160mm}
\caption{Cluster Masses}
\begin{tabular}{cccccccccc}
\label{tab:clustermassestimates}
Name & $N({\rm cores})$ & $N({\rm H\textsc{ii}})$ & $M_{\rm count}$ & $M_{\rm inferred}$ & $M_{\rm inferred, H\textsc{ii}}$ & $M_{\rm inferred, cores}$ & $M_{\rm count}^{\rm s}$ & $M_{\rm inf}^{\rm s}$ & SFR \\
 &  &  & $\mathrm{M_{\odot}}$ & $\mathrm{M_{\odot}}$ & $\mathrm{M_{\odot}}$ & $\mathrm{M_{\odot}}$ & $\mathrm{M_{\odot}}$ & $\mathrm{M_{\odot}}$ & $\mathrm{M_{\odot}\,yr^{-1}}$ \\
\hline
M & 17 & 47 & 2300 & 8800 & 15000 & 2300 & 1295 & 20700 & 0.012 \\
N & 11 & 3 & 270 & 1200 & 980 & 1500 & 150 & 2400 & 0.0017 \\
NE & 4 & 0 & 48 & 270 & 0 & 540 & 52 & 1200 & 0.00037 \\
S & 5 & 1 & 110 & 500 & 330 & 680 & 50 & 1100 & 0.00068 \\
Unassociated & 203 & 6 & 2700 & 15000 & 2000 & 27000 & - & - & 0.02 \\
Total & 240 & 57 & 5500 & 26000 & 19000 & 33000 & 1993 & 33400 & 0.035 \\
Total$_{\rm max}$ & - & - & - & 46000 & - & - & - & - & 0.062 \\
Clustered$_{\rm max}$ & - & - & - & 18000 & - & - & - & - & 0.024 \\
\hline
\end{tabular}\\
Reproduction and expansion of Table 2 in \citet{Ginsburg2018a}. $M_{\rm count}$ is the mass of directly counted protostars, assuming each millimeter source is 12.0 \msun, or 45.5 \msun if it is also an \hii region.  $M_{\rm inferred,cores}$ and $M_{\rm inferred,\hii}$ are the inferred total stellar masses assuming the counted objects represent fractions of the total mass 0.09 (cores) and 0.14 (\hii regions).  $M_{\rm inferred}$ is the average of these two.  $M_{\rm count}^{\rm s}$ and $M_{\rm inf}^{\rm s}$ are the counted and inferred masses reported in \citet{Schmiedeke2016a}.  The star formation rate is computed using $M_{\rm inferred}$ and an age $t=0.74$ Myr, which is the time of the last pericenter passage in the \citet{Kruijssen2015a} model.  The \emph{Total} row represents the total over the whole observed region.  The \emph{Total}$_{\rm max}$ row takes the higher of $M_{\rm inferred,\hii}$ and $M_{\rm inferred,cores}$ from each row and sums them.  We have included \hii regions in the $N(\hii)$ counts  that are \emph{not} included in our source table \citep[Table~3 of][]{Ginsburg2018a} because they are too diffuse, or because they are unresolved in our data but were resolved in the \citet{De-Pree2014a} VLA data.  As a result, the total source count is greater than the source count reported in Table~3 of \citet{Ginsburg2018a}. Also, the unassociated \hii region count is incomplete; it is missing both diffuse \hii regions and possibly unresolved hypercompact \hii regions, since there are no VLA observations comparable to \citet{De-Pree2014a} in the unassociated regions. {\bf There is too much irrelevant information in this table and the table caption is rambling too much. This needs to be trimmed down. Especially given that this is a letter (but also more broadly), every bit of information needs to be targeted at the point of the paper and it should be explained why it is given.}
\end{minipage}
\end{table*}


\subsection{Cluster membership in Sgr B2 M}
\label{sec:mmass}
\citet{Schmiedeke2016a} marked the Sgr B2 M cluster as a 13\arcsec  (0.5 pc) radius
region centered on Sgr B2 M F3.  Within this volume, there are 47 \hii regions
in the joint  \hii region catalogs \citep{Gaume1995a,De-Pree2015a} from their
0.05\arcsec resolution 7 mm Q-band
VLA observations.  There are 17 non-\hii-region cores, the faintest of which is
1.3 mJy at 3 mm \citep{Ginsburg2018a}.  By extrapolating the \hii region counts,
\citet{Ginsburg2018a} inferred a total stellar mass of 1.5\ee{4} \msun.
%somewhat lower than the 2\ee{4} \msun reported by \citet{Schmiedeke2016a};
%we adopt the .

% Column density calculation
% 40" = 1.6pc
% ((2.3e24*u.cm**-2) * (40*u.arcsec/2.35 * 8.5*u.kpc)**2 * 2*np.pi * 2.8*u.Da).to(u.Msun, u.dimensionless_angles())
% {\color{red} There is some inconsistency in \citet{Schmiedeke2016a} in these
% next two paragraphs.}
% 
% Based on \citet{Schmiedeke2016a}'s column density measurement of
% $N(\hh)=2.3\ee{24}~\persc$
% in a 40\arcsec beam toward Sgr B2 M,
% % this is from Table 3, using the 3D model
% the total gas mass in the center-most beam is $M_{gas, M} = 1.6\ee{5} \msun$.
% The instantaneous SFE is only $M_*/M_{gas}=1\%$.

% Table B.3. gives: 
% M2: ((2e6*u.cm**-3 * 2.8 * u.Da) * 4/3*(2e4*u.au)**3).to(u.M_sun)
% <Quantity 167.00206886275507 solMass>
% M1: ((2e6*u.cm**-3 * 2.8 * u.Da) * 4/3*(3e4*u.au)**3).to(u.M_sun)
% <Quantity 563.6319824117984 solMass>

% Table 2 gives:
\citet{Schmiedeke2016a} give a gas mass of $M=9.6\ee{3}$ \msun in the Sgr B2 M
cluster in their Table 2 and measure an instantaneous SFE of about 60\%, assuming
a cluster radius $r=0.5$ pc.  The total mass within 0.5 pc is then about $M_M =
2.5\ee{4}$ \msun, and the escape speed is $v_{esc}=14~\kms$.


% While it is possible that some of these sources are unassociated with Sgr B2,
% their proximity to the Sgr B2 core suggests they are indeed bound...

% mass measurement:
% (7e3*u.M_sun/u.pc**3 * (1.4*u.pc)**3 * 4/3.*np.pi)
It is possible the Sgr B2 M cluster is substantially larger, 35\arcsec (1.4 pc).
Within this radius, the `core' count is larger, 52 rather than 17, but the \hii
region count increases only marginally, from 47 to 49.  By contrast,
the gas mass is larger, $M_{gas,1.4pc} = 8\ee{4}~\msun$, so the integrated SFE is lower,
about 20\%.  
%Since the \hii region-inferred
%stellar mass is larger in both cases, we use this estimate, but
The presence of many cores in the outskirts of the Sgr B2 M cluster suggests
both that it may grow in stellar mass by accretion by up to an additional
$\sim50\%$ and that the lack of cores in the innermost region is due to
incompleteness (e.g., from confusion) rather than their absence, as suggested
in \citet{Ginsburg2018a}.

The assumed stellar mass  $M_M=1.5\ee{4}$ \msun is therefore a lower limit.

\subsection{Cluster membership in Sgr B2 N}
\citet{Schmiedeke2016a} marked the Sgr B2 N cluster as a 10\arcsec  (0.4 pc) radius circle
centered on Sgr B2 N K2.  \citet{Schmiedeke2016a} identified 3 compact \hii regions
and \citet{Ginsburg2018a} identified 11 cores within this region.  The inferred
total stellar plus protostellar mass is 980-1500 \msun.  However, unlike Sgr B2
M, Sgr B2 N is gas-dominated, with $M_{gas,N} = 2.8\ee{4}~\msun$ and SFE
$\sim5\%$ \citep{Schmiedeke2016a}.  The escape speed from the 0.4 pc cluster is
$v_{esc} = 18~\kms$.

Sgr B2 N is therefore better described as a `protocluster', in contrast with
Sgr B2 M, which is a (very) young massive cluster (YMC).  Sgr B2 N will need to
form an additional several thousand \msun of stars to form a YMC, and will need
to do so at high efficiency.  However, since there is evidence that the
protocluster itself is still rapidly accreting both stars and gas, this outcome
is quite likely.


\subsection{Velocity Dispersion Measurements - boundedness}
\label{sec:vdisp}
We measure the velocity dispersion of the stars as probed by their surrounding
\hii regions to confirm that the clusters are presently gravitationally bound.
Because these regions are mostly hypercompact \hii regions with radii less than a few
thousand AU, they closely follow the motions of their stars and serve as reasonable
probes of the underlying stellar kinematics.

We compare our velocity measurements to those of \citet{De-Pree2011a} and
\citet{De-Pree1996a} and perform  new velocity measurements based on the
data from \citet{Ginsburg2018a}.  Of the 32 unique \hii regions within the field
identified in the \citet{Gaume1995a} 1.3 cm data, 
%which have resolution comparable
%to the \citet{De-Pree2011a} H52$\alpha$ and H66$\alpha$ and the \citet{Ginsburg2018a}
%H41$\alpha$ data,
15 had measurements in \citet{De-Pree2011a}.  We have measured an additional 11
velocities from the H41$\alpha$ radio recombination line.  Our measurements
agree to within 5 \kms with those of \citet{De-Pree2011a} for all sources we
both measured except F10.37, for which we measure a $\sim20~\kms$ discrepancy;
our spectrum is of much higher quality, so we adopt the H41$\alpha$ measurement
as correct.  All measured velocities are reported in Table \ref{tab:h41afits}.

We measure the 1D velocity dispersion in Sgr B2 M by taking the standard
deviation of the measured V$_{LSR}$ values.  Using only the
\citet{De-Pree2011a} measurements, we obtain $\sigma_{1D}\approx9~\kms$.  Using
the full data set, we obtain a higher $\sigma_{1D}\approx12~\kms$.  In both
cases, $\sigma_{1D}$ is significantly lower than the escape velocity
$v_{esc}=14~\kms$ reported in Section \ref{sec:mmass}.
%1D Velocity Dispersion of 41a: 11.773405291227526, 52a: 9.013305133051228, 66a: 9.52127673916695

However, some individual sources are moving at high velocity with respect to
the average ($\bar{v}_{LSR}(H41\alpha) = 58.5~\kms$, $\bar{v}_{LSR}(H52\alpha)
= 65.8~\kms$), the fastest being G10.47 at $v_{LSR}=34$~\kms or
$v_{rel}=24-32$~\kms.  There is a small group at these highly negative
velocities and a projected distance from the center $r<0.1$ pc; these may be
bound to a deeper potential than we have inferred above, or they could be unbound
from the main cluster.
The \hii region J is separated by 0.4 pc and 16--24 \kms and is a diffuse \hii
region.  It may not be connected with the rest of the cluster.
If we exclude regions J, F10.37, G, G10.44, and G10.47, the velocity dispersion
drops to $\sigma_{1D}\approx8~\kms$.  If we exclude these sources, the total
inferred stellar mass for Sgr B2 M drops from $M_M = 1.5\ee{4}~\msun$ to
$M_M=1.3\ee{4}~\msun$.

% {\color{red} This paragraph hints at a result that may belong in a different paper.}
% The high velocity dispersion observed within this tiny region implies a 
% small dynamical timescale, 20-50 kyr for diameters 0.1--0.2 pc and velocity dispersions
% 8--12 \kms.  It is possible that the central cluster has undergone substantial
% dynamical evolution 

\clearpage
\begin{table}[htp]
\caption{H41$\alpha$ Line Fits}

\begin{tabular}{llllllllllllllllll}
\label{tab:h41afits}
Source & Coordinates & V$_\mathrm{LSR}$(41) & $\mathrm{FWHM}$(41) & V$_\mathrm{LSR}$(52) & $\mathrm{FWHM}$(52) & V$_\mathrm{LSR}$(66) & $\mathrm{FWHM}$(66) \\
 &  & $\mathrm{km\,s^{-1}}$ & $\mathrm{km\,s^{-1}}$ &  &  &  &  \\
\hline
A1 & 17:47:19.436 -28:23:01.36 & 63 & 32 & 60 & 34 & 64 & 36 \\
A2 & 17:47:19.566 -28:22:55.95 & 58 & 26 & - & - & - & - \\
B & 17:47:19.907 -28:23:02.91 & 76 & 35 & 72 & 38 & 71 & 41 \\
B10.06 & 17:47:19.868 -28:23:01.41 & 50 & 31 & - & - & 46 & 36 \\
B10.10 & 17:47:19.908 -28:23:02.13 & 70 & 27 & - & - & - & - \\
B9.96 & 17:47:19.776 -28:23:10.18 & 58 & 30 & - & - & - & - \\
B9.99 & 17:47:19.802 -28:23:06.9 & 62 & 23 & - & - & - & - \\
D & 17:47:20.053 -28:23:12.87 & 64 & 34 & 64 & 19 & 64 & 34 \\
E & 17:47:20.071 -28:23:08.65 & 61 & 29 & 63 & 26 & 62 & 35 \\
F1 & 17:47:20.12 -28:23:04.26 & 81 & 72 & 85 & 34 & 79 & 61 \\
F10.303 & 17:47:20.112 -28:23:03.7 & 57 & 81 & 57 & 57 & - & - \\
F10.33 & 17:47:20.14 -28:23:06.1 & 55 & 36 & - & - & - & - \\
F10.35 & 17:47:20.156 -28:23:06.73 & 59 & 72 & - & - & - & - \\
F10.37 & 17:47:20.179 -28:23:05.95 & 40 & 58 & 60 & 16 & - & - \\
F10.39 & 17:47:20.195 -28:23:06.65 & 63 & 53 & - & - & - & - \\
F2 & 17:47:20.17 -28:23:03.75 & 78 & 98 & 75 & 38 & 68 & 78 \\
F3 & 17:47:20.176 -28:23:04.81 & 61 & 45 & 61 & 50 & 68 & 63 \\
F4 & 17:47:20.219 -28:23:04.34 & 67 & 43 & 70 & 39 & 72 & 59 \\
G & 17:47:20.287 -28:23:03.07 & 44 & 44 & 49 & 42 & 47 & 44 \\
G10.44 & 17:47:20.246 -28:23:03.36 & 40 & 33 & - & - & - & - \\
G10.47 & 17:47:20.274 -28:23:02.38 & 34 & 22 & - & - & - & - \\
I & 17:47:20.511 -28:23:06.08 & 60 & 31 & 65 & 27 & 61 & 37 \\
I10.52 & 17:47:20.329 -28:23:08.14 & 61 & 23 & - & - & - & - \\
J & 17:47:20.574 -28:22:56.17 & 43 & 28 & - & - & - & - \\
\hline
\end{tabular}
\par
CAPTION GOES HERE
\end{table}

\clearpage

\subsection{The cluster formation efficiency}
Table \ref{tab:clustermassestimates} shows the breakdown of ongoing star
formation within the Sgr B2 region.  The total inferred mass of recently
formed or forming stars is ${M_{*,total}\approx4.6\ee{4}~\msun}$ spread across
the whole cloud, with ${M_{*,clustered}\approx1.7\ee{4}~\msun}$ concentrated
in the Sgr B2 M and N clusters.  These values imply a \textit{cluster
formation efficiency} ${\Gamma=M_{*,clustered}/M_{*,total} = 37\%}$.

We have noted above that the membership of clusters M and N could be expanded,
and while this expansion would have no effect on the estimated mass of the clusters
(because their masses have been inferred from more complete samples of \hii regions),
it would reduce the number of unassociated cores by about 25\%, increasing the inferred
CFE to $\approx43\%$. % this number calculated by hand: m_tot = 46k - (27k*0.25) = 40, 17 / 40 = 43%.
% Treating Sgr B2 NE and S as clusters would also serve to increase our CFE estimates,
% but only by 2-3\%, so we ignore them.

\subsection{Comparison of observations to predictions}
\citet{Kruijssen2012a} described a theoretical framework in which the fraction
of stars forming in bound clusters ($f_{bound}$) can be predicted based on both
global Galactic-scale and local physical conditions.  Using observed parameters
described in Table \ref{tab:model}, we obtain predictions for $f_{bound}$ and
compare them to the observed values of 37-43\%.  The model predictions are
slightly higher than the observed values, but they are consistent within the expected
errors (see Table \ref{tab:model}).

Figure \ref{fig:figure} shows the comparison of Sgr B2 with other data and with
the theoretical prediction of the \citet{Kruijssen2012a} model.  The comparison
data sets are a compilation from \citet{Adamo2015a}, plus spatially resolved
M83 data from \citet{Freeman2017a} and \citet{Adamo2015a}.

{\color{red}Do we need this paragraph and is it truly appropriate?  Might it be
simpler to say ``We omit the CCE because these clusters are still in the
collapse phase." ?  At least, the middle two sentences of this paragraph do not
seem relevant to \emph{this} data set because they discuss issues related to
the future evolution of the clusters.}
\par{\color{blue} 
When evaluating the CFE, we only consider the bound fraction of star formation
and omit the `cruel cradle effect' (CCE), i.e., the tidal disruption of
star-forming overdensities by dense substructure in the ISM. This choice is
made because the clouds on the CMZ dust ridge are moving coherently on a thin
stream, which limits the encounter rate relative to galactic discs.  The CCE
computed in \citet{Kruijssen2012a} assumed encounters with clouds could occur
in all three dimensions, so the timescales used in that model are not
appropriate for CMZ clouds.  Since we observe Sgr B2 at a very early stage, the
clusters there have likely experienced few, if any, molecular cloud encounters
driving disruptive tidal shocks.
}

The first panel in Figure \ref{fig:figure} shows the CFE $\Gamma$ as a function
of gas surface density.  The data from this work on Sgr B2 are shown as an
orange point, and the predictions from the \citet{Kruijssen2012a} model are
overlaid in red and blue for the global and local versions of that theory,
respectively.  The model and data are clearly in agreement.  The first two
panels show that Sgr B2 fits along the relations defined by observations of
other galaxies.  The third panel is included to illustrate that the distance
to the object is not a predictor of the CFE (a concern raised
by \citealt{Adamo2011a} for extragalactic samples); instead, a range of CFEs can be
seen even within our own Galaxy.

%\FigureThreePDF
%{GammaVsDistance}
%{GammaVsSigmaSFR}
%{GammaVsSigmaGas_withTheory}
\Figure{GammaVsEverything_3panel.pdf}
{Panel 1 shows the CFE as a function of distance, demonstrating that there is
contrast in the CFE even within our Galaxy.
Panel 2 shows the CFE vs SFR.
In panel 3, contours are shown from evaluating the \citet{Kruijssen2012a} model
over the parameter space indicated in Table \ref{tab:model}.  Contours are shown
at the 95\% and 68\% confidence levels for the local (blue) and global (red).
}
{fig:figure}{1}{\textwidth}

\section{Conclusions}
We have measured the cluster formation efficiency in the Galactic center cloud
Sgr B2, $\Gamma=37-43\%$.  The CFE model of \citet{Kruijssen2012a} predicts the
observed value to within a factor of 2.

\begin{table*}[htp]
\centering
\caption{Model parameters}
\begin{tabular}{ccccccc}
\label{tab:model}
Quantity & Units & Median & Uncertainty & `Global model' & `Local model' & Reference \\
\hline
$\log{\Sigma}$ & [$\msun~\pc^{-2}$] & 3.00 & 0.22 & \checkmark &  & \citenum{Henshaw2016b} \\
$\Omega$ & [$\myr^{-1}$] & 1.80 & 0.25 & \checkmark &  & \citenum{Launhardt2002}, \citenum{Kruijssen2015a} \\
$\log{\rho}$ & [$\msun~\pc^{-3}$] & 2.84 & 0.22 &  & \checkmark & \citenum{Longmore2013} \\
$c_{\rm s}$ & [$\kms$] & 0.53 & 0.07 &  & \checkmark & \citenum{Ginsburg2016}, \citenum{Krieger2018} \\
$\log{\sigma}$ & [$\kms$] & 1.00 & 0.07 & \checkmark & \checkmark & \citenum{Kruijssen2018} \\
$\log{\Sigma_{\rm GMC}}$ & [$\msun~\pc^{-2}$] & 3.61 & 0.18 & \checkmark & \checkmark & \citenum{Walker2015}, \citenum{Federrath2016} \\
$\log{\alpha_{\rm vir}}$ & [--] & 0.04 & 0.16 & \checkmark & \checkmark & \citenum{Walker2015} \\
$\log{\beta_0}$ & [--] & $-0.47$ & 0.32 & \checkmark & \checkmark & \citenum{Federrath2016} \\
$t_{\rm view}$ & [$\myr$] & 0.74 & 0.16 & \checkmark & \checkmark & \citenum{Kruijssen2015a} \\
\hline

$f_{bound,global}$ & & 44.5\% & $^{+13.3}_{-13.0}$ & - & - & \citenum{Kruijssen2012a} \\
$f_{bound,local}$ & & 48.7\% & $^{+10.8}_{-12.2}$ & - & - & \citenum{Kruijssen2012a} \\
\end{tabular}
\par
\tablerefs{
(\citenum{Barnes2017}) \citealt{Barnes2017}, %2017MNRAS.469.2263B
(\citenum{Federrath2016}) \citealt{Federrath2016}, %2016ApJ...832..143F
(\citenum{Ginsburg2016}) \citealt{Ginsburg2016}, %2016A&A...586A..50G
(\citenum{Henshaw2016b}) \citealt{Henshaw2016b}, %http://adsabs.harvard.edu/abs/2016MNRAS.463L.122H
(\citenum{Krieger2017}) \citealt{Krieger2017}, %2017ApJ...850...77K
(\citenum{Kruijssen2015a}) \citealt{Kruijssen2015a}, %
(\citenum{Kruijssen2018}) \citealt{Kruijssen2018}, %
(\citenum{Launhardt2002}) \citealt{Launhardt2002}, %2002A&A...384..112L
(\citenum{Longmore2013}) \citealt{Longmore2013}, %2013MNRAS.429..987L
(\citenum{Walker2015}) \citealt{Walker2015}. %2015MNRAS.449..715W
}
\end{table*}



\bibliographystyle{aasjournal}
\bibliography{extracted}

\end{document}
