\begin{table*}[htp]
\centering
\begin{minipage}{130mm}
\caption{Cluster Masses}
\begin{tabular}{cccccc}
\label{tab:clustermassestimates}
Name & $N({\rm cores})$ & $N({\rm H\textsc{ii}})$ & $M_{\rm inferred, cores}$ & $M_{\rm inferred, H\textsc{ii}}$ & $M_{\rm inferred,max}$ \\
 &  &  & $\mathrm{M_{\odot}}$ & $\mathrm{M_{\odot}}$ & $\mathrm{M_{\odot}}$ \\
\hline
M & 17 & 47 & 2300 & 15000 & 15000 \\
N & 11 & 3 & 1500 & 980 & 1500 \\
NE & 4 & 0 & 540 & 0 & 540 \\
S & 5 & 1 & 680 & 330 & 680 \\
Unassociated & 203 & 6 & 27000 & 2000 & 27000 \\
Total & 240 & 57 & 33000 & 19000 & 46000 \\
Clustered with NE, S & 37 & 51 & 5000 & 17000 & 18000 \\
Clustered only M, N & 28 & 50 & 3800 & 16000 & 17000 \\
\hline
\end{tabular}
\\
Partial reproduction of Table 2 in \citet{Ginsburg2018a}. $M_{\rm inferred,cores}$ and $M_{\rm inferred,\hii}$ are the inferred total stellar masses assuming the counted objects represent fractions of the total mass of 0.09 (cores) and 0.14 (\hii regions).  $M_{\rm inferred,max}$ is the greater of these two.  The \emph{Total} row represents the total over the whole observed region.  The two \emph{Clustered} rows show the total inferred mass of clusters including all four candidate clusters M, N, NE and S, then the mass of clusters including only M and N.
\end{minipage}
\end{table*}
